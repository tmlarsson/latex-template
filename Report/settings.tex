%%%%%%%%%%%%%%%%%%%%%%%%%%%%%% Preamble %%%%%%%%%%%%%%%%%%%%%%%%%%%%%%
\documentclass[a4paper, 12pt]{article}

\usepackage[T1]{fontenc}
\usepackage[utf8]{luainputenc}
\usepackage[british]{babel}
\usepackage[titles]{tocloft}            % Better table of contents
>>>>>>> upstream/master
\usepackage[headheight=35pt]{geometry}  % Page geometry
\usepackage{graphicx,
            epstopdf,
            float,
            subcaption}                 % Figures
<<<<<<< HEAD
=======
\usepackage[margin=3ex,
            font=small,
            labelfont=bf,
            labelsep=endash]{caption}
>>>>>>> upstream/master
\usepackage[section]{placeins}          % Gives \FloatBarrier
\usepackage{amsmath,
            amssymb,
            amsthm,
            amsfonts,
            mathrsfs,
            dsfont,
            esint,
            mathtools,
            tensor,
            thmtools}                     % Mathematics
\usepackage{physics,
            siunitx}                    % Physics
\usepackage{enumerate}                  % Lists
\usepackage{fancyhdr}                   % Header and footer
\usepackage{booktabs,
            multirow}                   % Tables
\usepackage{appendix}                   % Appendices
\usepackage{hyperref}                   % Links
\usepackage[capitalise,
            noabbrev]{cleveref}         % Automatic references
\usepackage{csquotes}                   % Quotes. For biblatex
\usepackage{color}                      % Colour
\usepackage{pdflscape}                  % Rotate the whole document
\usepackage{multicol}                   % Multiple columns on one page
\usepackage{pgfplots}                   % Create figures
\usepackage{lmodern}                    % More options for fonts
\usepackage{newfloat}                   % For minted
\usepackage{parskip}			% Begin on new row
%\usepackage[newfloat]{minted}          % Insert code
%\usepackage[backend=biber,
%           style=ieee,
%           language=auto,
%           sortcites=true,
%           url=true,
%           hyperref=true]{biblatex}

%%%%%%%%%%%
% Counters efter section

\numberwithin{equation}{section}
\numberwithin{figure}{section}
\numberwithin{table}{section}

\makeatletter
    \newtoggle{biblatexloaded}
    \newtoggle{mintedloaded}
    \@ifpackageloaded{biblatex}{\toggletrue{biblatexloaded}}{\togglefalse{biblatexloaded}}
    \@ifpackageloaded{minted}{\toggletrue{mintedloaded}}{\togglefalse{mintedloaded}}
\makeatother

%% Cleveref
\crefname{listing}{Code}{Codes} % Match with minted caption

%% Pgfplots
\pgfplotsset{compat=newest,scaled ticks = false,
            tick label style={/pgf/number format/fixed},
            /pgf/number format/1000 sep={}}

\usetikzlibrary{spy}
%\usetikzlibrary{external}
%\tikzexternalize
%\tikzsetexternalprefix{tikz/}

%% Tocloft
% Set the name of figures and tables in the list of figures and tables
\newlength{\fignamelength}
\settowidth{\fignamelength}{\tablename}
\addtolength{\cftfignumwidth}{\fignamelength}
\renewcommand{\cftfigpresnum}{\figurename~}

\newlength{\tabnamelength}
\settowidth{\tabnamelength}{\figurename}
\addtolength{\cfttabnumwidth}{\tabnamelength}
\renewcommand{\cfttabpresnum}{\tablename~}

%% Siunitx
\sisetup{exponent-product=\cdot, % Use \cdot instead of \times as replacement for 10-exponentials
        output-complex-root=\ensuremath{i}, % Normal math-i instead of \mathrm{i},
        group-separator = {}} % Don't separate thousands

%% Biblatex
%%%%%%%%%%%%%%%%%%%%%%%%%%%%%% Settings %%%%%%%%%%%%%%%%%%%%%%%%%%%%%%

%%%%%%%%%%%%%%% Biblatex %%%%%%%%%%%%%%%
\iftoggle{biblatexloaded}{
    \addbibresource{refs.bib} % File with bibliography database
    % Surname first
    \DeclareNameAlias{sortname}{last-first}
    \DeclareNameAlias{default}{last-first}

    \urlstyle{sf}
}

%%%%%%%%%%%%%%% Cleveref %%%%%%%%%%%%%%%
\crefname{listing}{Code}{Codes} % Match with minted caption

%%%%%%%%%%%%%%% Minted %%%%%%%%%%%%%%%
\iftoggle{mintedloaded}{
    % C
    \newmintedfile{c}{fontfamily=tt,
        fontsize=\footnotesize,
        tabsize=4,
        numberblanklines=true,
        numbers=left,
        numbersep=5pt,
        breakautoindent=false,
        xleftmargin=0.7cm,} % \cfile{}

    \newminted[ccodecap]{c}{fontfamily=tt,
        fontsize=\normalsize,
        tabsize=4,
        frame=lines,
        breaklines=true,
        breaksymbolleft=\tiny\ensuremath{\hookrightarrow},
        breakautoindent=true,} %\begin{ccodecap}

    \newmintinline{c}{fontfamily=tt,
        breaklines=true,
        fontsize=\normalsize} %\cinline{}

    % Bash
    \newminted[bashcode]{bash}{fontfamily=tt,
        fontsize=\normalsize,
        tabsize=4,} %\begin{bashcode}

    \newminted[bashcodecap]{bash}{fontfamily=tt,
        fontsize=\normalsize,
        tabsize=4,
        frame=lines,
        breaklines=true,
        breaksymbolleft=\tiny\ensuremath{\hookrightarrow},
        breakautoindent=true} %\begin{bashcodecap}

    \newmint[bash]{bash}{fontfamily=tt,
        fontsize=\normalsize,
        frame=lines,
        breaklines=true,
        breaksymbolleft=\tiny\ensuremath{\hookrightarrow},
        breakautoindent=true} % \bash{}

    \newmintedfile[bashfile]{bash}{fontfamily=tt,
        fontsize=\normalsize,
        tabsize=4,
        breakautoindent=false,} % \bashfile{}

    \newmintinline[bashinline]{bash}{fontfamily=tt,
        fontsize=\normalsize,
        breaklines=true,
        breakautoindent=false} % \bashinline{}

    % MATLAB
    \newmintedfile{matlab}{fontfamily=tt,
        fontsize=\footnotesize,
        tabsize=4,
        numberblanklines=true,
        numbers=left,
        numbersep=5pt,
        breakautoindent=false,
        xleftmargin=0.7cm,} % \matlabfile{}

    \newmintedfile[matlabfileframe]{matlab}{fontfamily=tt,
        fontsize=\footnotesize,
        tabsize=4,
        frame=lines,
        numberblanklines=true,
        numbers=left,
        numbersep=5pt,
        breakautoindent=false,
        xleftmargin=0.7cm,} % \matlabfileframe{}

    \newminted[mcodecap]{matlab}{fontfamily=tt,
        fontsize=\normalsize,
        tabsize=4,
        frame=lines,
        breaklines=true,
        breaksymbolleft=\tiny\ensuremath{\hookrightarrow},
        breakautoindent=true,} %\begin{mcodecap}

    \newminted[mcode]{matlab}{fontfamily=tt,
        tabsize=4,
        breaklines=true,
        breaksymbolleft=\tiny\ensuremath{\hookrightarrow},
        breakautoindent=true,} %\begin{mcode}

    \newmintinline[minline]{matlab}{fontfamily=tt,
        breaklines=true,
        fontsize=\normalsize} %\minline{}

    \SetupFloatingEnvironment{listing}{name=Code} % Shows "Code" in captions
}

%%%%%%%%%%%%%%% Pgfplots %%%%%%%%%%%%%%%
\pgfplotsset{compat=newest,scaled ticks = false,
    tick label style={/pgf/number format/fixed},
    /pgf/number format/1000 sep={}}

\usetikzlibrary{spy}
%\usetikzlibrary{external}
%\tikzexternalize
%\tikzsetexternalprefix{tikz/}

%%%%%%%%%%%%%%% Siunitx %%%%%%%%%%%%%%%
\sisetup{exponent-product=\cdot, % Use \cdot instead of \times as replacement for 10-exponentials
    output-complex-root=\ensuremath{\cplex},
    group-separator = {}} % Don't separate thousands

%%%%%%%%%%%%%%% Tocloft %%%%%%%%%%%%%%%
% Set the name of figures and tables in the list of figures and tables
\newlength{\fignamelength}
\settowidth{\fignamelength}{\tablename}
\addtolength{\cftfignumwidth}{\fignamelength}
\renewcommand{\cftfigpresnum}{\figurename~}

\newlength{\tabnamelength}
\settowidth{\tabnamelength}{\figurename}
\addtolength{\cfttabnumwidth}{\tabnamelength}
\renewcommand{\cfttabpresnum}{\tablename~}

%%%%%%%%%%%%%%% Misc %%%%%%%%%%%%%%%
% Links
\hypersetup{colorlinks=true,
    hidelinks}
			
% Definitions of commands
\newcommand{\mail}[1]{\href{mailto:#1}{\nolinkurl{#1}}} % Email addresses as links
\newcommand{\pd}[0]{\partial} % Partial differential d
\newcommand{\matr}[1]{#1} % Matrices
\newcommand{\trps}[0]{\mathrm{T}} % Transpose of vector or matrices
\DeclareMathOperator{\linspan}{span} % Span of spaces in linear algebra
\newtheorem{theorem}{Theorem} % \begin{theorem}[Name of Theorem]
\newcommand{\me}[1]{\mathrm{e}^{#1}} % Exponential e
\newcommand{\cplex}[0]{\mathrm{i}} % Imaginary i

% Redefine \left and \right in order to make them look nicer when used in functions
\let\originalleft\left
\let\originalright\right
\renewcommand{\left}{\mathopen{}\mathclose\bgroup\originalleft}
\renewcommand{\right}{\aftergroup\egroup\originalright}

% Folder with figures
\graphicspath{./figs/}





% Jespers settings

\newcommand{\vect}[1]{\boldsymbol{#1}}
\renewcommand{\emptyset}{\varnothing}

\declaretheoremstyle[%
    spaceabove = 6pt,
    spacebelow = 6pt,
    headfont = \bfseries,
    bodyfont = \itshape,
]{sats}

\declaretheoremstyle[%
    spaceabove = 6pt,
    spacebelow = 6pt,
    headfont = \bfseries,
    bodyfont = \normalfont,
]{def}

\declaretheoremstyle[%
    spaceabove = 6pt,
    spacebelow = 6pt,
    headfont = \itshape,
    bodyfont = \normalfont,
]{anm}

\theoremstyle{sats}
\newtheorem{sats}{Sats}
\newtheorem{lemma}[sats]{Lemma}
\newtheorem{prop}[sats]{Proposition}
\newtheorem{foljdsats}[sats]{Följdsats}
\theoremstyle{def}
\newtheorem{definition}[sats]{Definition}
\newtheorem{exempel}[sats]{Exempel}
\theoremstyle{anm}
\newtheorem{anm}{Anmärkning}
\newtheorem*{anm*}{Anmärkning}

\DeclareMathOperator{\vol}{Vol}
\DeclareMathOperator{\diam}{diam}
\DeclareMathOperator{\dist}{dist}
\DeclareMathOperator{\supp}{supp}

\newcommand{\REP}[1]{\mbox{$\mathbb{R}^{#1}$}}
\newcommand{\RE}{\mbox{$\mathbb{R}$}}
\newcommand{\NA}{\mbox{$\mathbb{N}$}}
\newcommand{\Q}{\mbox{$\mathbb{Q}$}}
\newcommand{\CEP}[1]{\mbox{$\mathbb{C}^{#1}$}}
\newcommand{\C}{\mbox{$\mathbb{C}$}}
\newcommand{\N}{\mbox{$\mathbb{N}$}}
\newcommand{\A}{\mbox{$\mathscr{A}$}}
\newcommand{\Cs}{\mbox{$\mathscr{C}$}}
\newcommand{\F}{\mbox{$\mathscr{F}$}}
\newcommand{\B}{\mbox{$\mathscr{B}$}}
\newcommand{\M}{\mbox{$\mathscr{M}$}}
\newcommand{\K}{\mbox{$\mathscr{K}$}}
\newcommand{\E}{\mbox{$\mathscr{E}$}}
\renewcommand{\L}{\mbox{$\mathscr{L}$}}
\newcommand{\Hs}{\mbox{$\mathscr{H}$}}
\renewcommand{\P}{\mbox{$\mathscr{P}$}}
\newcommand{\epr}{\begin{flushright}\mbox{$\square$}\end{flushright}}
\newcommand{\e}{\mbox{$\mathrm{e}$}}
